\documentclass{report}

\title{Notes for General Topology}
\author{Wenchuan Zhao}
\usepackage[toc,page]{appendix}


\usepackage[utf8]{inputenc}
\usepackage{amsfonts}
\usepackage{amsthm} %proof
\usepackage{amssymb} %subsetneq
\usepackage{amsmath} %aligned
\usepackage{enumerate}
\usepackage{mathrsfs} %for `\mathscr`


\usepackage{graphicx}
\graphicspath{ {./images/} }

\usepackage{layouts}
%textwidth in cm: \printinunitsof{pt}\prntlen{\textwidth}


\newtheoremstyle{wenchuan}
	{1.25em} % Space above
	{1.25em} % Space below
	{} % Body font
	{} % Indent amount
	{\bfseries} % Theorem head font
	{.} % Punctuation after theorem head
	{0.5em} % Space after theorem head
	{} % Theorem head spec (can be left empty, meaning `normal')


%\theoremstyle{definition}

\theoremstyle{wenchuan}
\newtheorem{definition}{Definition}[section]
\newtheorem{proposition}{Proposition}[section]
\newtheorem{observation}{Observation}[section]
\newtheorem{problem}{Problem}[section]

\newtheorem{theorem}{Theorem}[section]
\newtheorem{corollary}{Corollary}[section]
\newtheorem{lemma}{Lemma}[section]

\newtheorem{example}{Example}[section]
\newtheorem{note}{Note}[section]


%\renewenvironment{proof}{{\bfseries Proof.}}{}


\newcommand{\qedlm}{\hfill \ensuremath{\Box}}
\renewcommand{\qed}{\hfill \ensuremath{\blacksquare}}

\renewcommand{\baselinestretch}{1.25}


% Colors
%================================================
%::::::::::::::::::::::::::::::::::::::::::::::::

\usepackage{hyperref}

%------------------------------------------------

\hypersetup{
    colorlinks=true,
    linkcolor=black,
    filecolor=black,      
    urlcolor=black,
}

%------------------------------------------------

%::::::::::::::::::::::::::::::::::::::::::::::::
%================================================



% Colors
%================================================
%::::::::::::::::::::::::::::::::::::::::::::::::

\usepackage{xcolor}

%------------------------------------------------

\definecolor{red}{HTML}{DF384F}
\definecolor{blue}{HTML}{0940DE}

%------------------------------------------------

%::::::::::::::::::::::::::::::::::::::::::::::::
%================================================



% Title Formatting
%================================================
%::::::::::::::::::::::::::::::::::::::::::::::::

\usepackage{titlesec}

%------------------------------------------------
\titleformat
{\chapter} % command
[display] % shape
{\bfseries\Large\itshape} % format
{\centering Chapter \ \thechapter.} % label
{0ex} % sep
{
%    \rule{\textwidth}{1pt}\vspace{1ex}
%    \vspace{1ex}
	\huge
    \centering
} % before-code
[
%\vspace{-0.3ex}%
%\rule{\textwidth}{0.3pt}
] % after-code
%------------------------------------------------

\titleformat
{\section}
[display]
{\bfseries\large}
{} % label
{0ex} % sep
{
	\S \thesection \ \
} % before-code
[
\vspace{-0.5ex}%
\rule{\textwidth}{0.3pt}\vspace{2ex}
] % after-code

%::::::::::::::::::::::::::::::::::::::::::::::::
%================================================



% Footnote
%================================================
%::::::::::::::::::::::::::::::::::::::::::::::::
\renewcommand{\footnoterule}{
  \kern 2ex
  \hrule width 160pt height 0.3pt
  \kern 2ex
}
%::::::::::::::::::::::::::::::::::::::::::::::::
%================================================


\begin{document}
\maketitle
\tableofcontents

%========================================
%########################################
%----------------------------------------



\chapter{Topological Space}
%========================================

\section{Basic Ideas}
%----------------------------------------


\begin{definition}
	\label{def: topology}
	
	Let $X$ be any set.
	
	A collection $\mathcal T \subseteq \mathcal P(X)$ is a \textit{topology} for $X$ iff it satisfies the \textit{open set axioms}:
	\begin{enumerate}
		\item
		$X \in \mathcal T$;
		
		\item
		$\mathcal T$ is closed under arbitrary union; explicitly,
		$$
		\forall \mathcal A \subseteq \mathcal T: \bigcup \mathcal A \in \mathcal T;
		$$
		
		\item
		$\mathcal T$ is closed under finite intersection; explicitly,
		$$
		\forall \mathcal B \subseteq \mathcal T : |\mathcal B| \in \mathbb N : \bigcap \mathcal B \in \mathcal T.
		$$
		
		The ordered pair $(X, \mathcal T)$ is a \textit{topological space} iff $\mathcal T$ is a topology for $X$.
		
		A subset $U \subseteq X$ is an \textit{open set} of $(X, \mathcal T)$, or an \textit{open subset} of $X$, iff $U \in \mathcal T$.
	\end{enumerate}
\end{definition}


\begin{note}
	Even if $\mathcal T$ is an infinite	topology on an infinite set $X$, $\mathcal T$ is not needed to be closed under infinite intersection. For example, let $\mathcal T$
	$$
	\mathcal T = \left\{ [0,r): r \in \mathbb R \right\}.
	$$
	then $\mathcal T$ is a topology for $\mathbb R_{\ge 0}$. The collection
	$$
	\left\{ \left[ 0, \frac{1}{i} \right) \right\}_{i \in \mathbb Z_{> 0}}
	$$
	is a subset of $\mathcal T$, but its intersection is $\{0\} \notin \mathcal T$.
\end{note}



\begin{lemma}
	\label{lm: emptyset is in topology}
	
	Let $(X, \mathcal T)$ be a topological space.
	
	Then, $\emptyset \in \mathcal T$.
\end{lemma}


\begin{proof}
	As $\emptyset$ is a subset of any set, $\emptyset \subseteq \mathcal T$. By the open set axiom 2, we have
	$$
	\emptyset = \bigcup \emptyset \in \mathcal T.
	$$
\end{proof}


\begin{definition}
	Let $X$ be any set, and let $\mathcal T_1$ and $\mathcal T_2$ be topologies on $X$.
	
	$\mathcal T_1$ is said to be \textit{finer} than $\mathcal T_2$, or $\mathcal T_2$ is said to be \textit{coarser} than $\mathcal T_1$ iff $\mathcal T_2 \subseteq \mathcal T_1$.
\end{definition}


\begin{example}
	For any set $X$, the power set $\mathcal P(X)$ can be considered as a topology for $X$, called \textit{discrete topology}. It is the \textit{finest topology} on $X$.
\end{example}

\begin{example}
	For any set $X$, the collection $\{\emptyset, X\}$ is a topology for $X$. It is called \textit{indiscrete topology}, or \textit{trivial topology}, which is the coarsest topology on $X$.
\end{example}


\section{Bases for Topologies}
%----------------------------------------


\begin{definition}
	Let $(X, \mathcal T)$ be a topological space.
	
	A collection $\mathcal B \subseteq \mathcal T$ is an \textit{analytic basis} for $\mathcal T$ iff for any $U \in \mathcal T$, there is a $\mathcal S \subseteq \mathcal B$, such that
	$$
	U = \bigcup \mathcal S.
	$$
\end{definition}














%----------------------------------------
%########################################
%========================================
\end{document}