\documentclass{report}

\title{Notes on General Topology}
\author{Wenchuan Zhao}
\usepackage[toc,page]{appendix}


\usepackage[utf8]{inputenc}
\usepackage{amsfonts}
\usepackage{amsthm} %proof
\usepackage{amssymb} %subsetneq
\usepackage{amsmath} %aligned
\usepackage{enumerate}
\usepackage{mathrsfs} %for `\mathscr`


\usepackage{graphicx}
\graphicspath{ {./images/} }

\usepackage{layouts}
%textwidth in cm: \printinunitsof{pt}\prntlen{\textwidth}


\newtheoremstyle{wenchuan}
	{1.25em} % Space above
	{1.25em} % Space below
	{} % Body font
	{} % Indent amount
	{\bfseries} % Theorem head font
	{.} % Punctuation after theorem head
	{0.5em} % Space after theorem head
	{} % Theorem head spec (can be left empty, meaning `normal')


%\theoremstyle{definition}

\theoremstyle{wenchuan}
\newtheorem{definition}{Definition}[section]
\newtheorem{proposition}{Proposition}[section]
\newtheorem{observation}{Observation}[section]
\newtheorem{problem}{Problem}[section]

\newtheorem{theorem}{Theorem}[section]
\newtheorem{corollary}{Corollary}[section]
\newtheorem{lemma}{Lemma}[section]

\newtheorem{example}{Example}[section]
\newtheorem{note}{Note}[section]


%\renewenvironment{proof}{{\bfseries Proof.}}{}


\newcommand{\qedlm}{\hfill \ensuremath{\Box}}
\renewcommand{\qed}{\hfill \ensuremath{\blacksquare}}

\renewcommand{\baselinestretch}{1.25}


% Colors
%================================================
%::::::::::::::::::::::::::::::::::::::::::::::::

\usepackage{hyperref}

%------------------------------------------------

\hypersetup{
    colorlinks=true,
    linkcolor=black,
    filecolor=black,      
    urlcolor=black,
}

%------------------------------------------------

%::::::::::::::::::::::::::::::::::::::::::::::::
%================================================



% Colors
%================================================
%::::::::::::::::::::::::::::::::::::::::::::::::

\usepackage{xcolor}

%------------------------------------------------

\definecolor{red}{HTML}{DF384F}
\definecolor{blue}{HTML}{0940DE}

%------------------------------------------------

%::::::::::::::::::::::::::::::::::::::::::::::::
%================================================



% Title Formatting
%================================================
%::::::::::::::::::::::::::::::::::::::::::::::::

\usepackage{titlesec}

%------------------------------------------------
\titleformat
{\chapter} % command
[display] % shape
{\bfseries\Large\itshape} % format
{\centering Chapter \ \thechapter.} % label
{0ex} % sep
{
%    \rule{\textwidth}{1pt}\vspace{1ex}
%    \vspace{1ex}
	\huge
    \centering
} % before-code
[
%\vspace{-0.3ex}%
%\rule{\textwidth}{0.3pt}
] % after-code
%------------------------------------------------

\titleformat
{\section}
[display]
{\bfseries\large}
{} % label
{0ex} % sep
{
	\S \thesection \ \
} % before-code
[
\vspace{-0.5ex}%
\rule{\textwidth}{0.3pt}\vspace{2ex}
] % after-code

%::::::::::::::::::::::::::::::::::::::::::::::::
%================================================



% Footnote
%================================================
%::::::::::::::::::::::::::::::::::::::::::::::::
\renewcommand{\footnoterule}{
  \kern 2ex
  \hrule width 160pt height 0.3pt
  \kern 2ex
}
%::::::::::::::::::::::::::::::::::::::::::::::::
%================================================


\begin{document}
\maketitle
\tableofcontents

%========================================
%########################################
%----------------------------------------



\chapter{Topological Spaces}
%========================================


\section{Metric Spaces}
%----------------------------------------


How do we measure the distance between two points in the space? Well, in the intuitive world, Pythagoras theorem might be the most popular way to do so.

Take the 2-dimensional Euclidean vector space $\mathbb R^2$ for example -- Here we use the $xy$-plane. For any vector $\mathbf a, \mathbf b \in \mathbb R^2$, their distance is defined as the length of segment with $\mathbf a$ and $\mathbf b$ as the end points. So, by Pythagoras theorem, it is by this formula:
$$
\rho(\mathbf a, \mathbf b) = \sqrt{(a_x - b_x)^2 + (a_y - b_y)^2}.
$$

Here, we consider $\rho$ as an operation defined on $\mathbb R^2 \times \mathbb R^2$, called metric function, and its outputs are real numbers:
$$
\rho: \mathbb R^2 \times \mathbb R^2 \to \mathbb R.
$$

In this case, $\rho$ satisfies 4 conditions: (let me write down the precondition first) for any $\mathbf a, \mathbf b, \mathbf c \in \mathbb R^2$
\begin{enumerate}
	\item $\rho(\mathbf a, \mathbf b) \ge 0$. Obviously, there is no negative distance.
	\item $\rho(\mathbf a, \mathbf b) = 0 \iff \mathbf a = \mathbf b$. That is, the distance between two points is zero if and only if they coincides.
	\item $\rho(\mathbf a, \mathbf b) = \rho(\mathbf b, \mathbf a)$. That is obvious, because a segment $\mathbf a \mathbf b$ is actually the same as $\mathbf b \mathbf a$.
	\item $\rho(\mathbf a, \mathbf b) + \rho(\mathbf b, \mathbf c) \ge \rho(\mathbf a, \mathbf c)$. We call this property the triangle inequality. It is a quite geometrical name, because if $\mathbf a, \mathbf b$ and $\mathbf c$ form a triangle, then the sum of the length of any two sides of the triangle can't be less than the length of the third one.
\end{enumerate}

So, in this case, we call the order pair $(\mathbb R^2, \rho)$ the 2-dimensional Euclidean metric space.

But, just like how we define the mass of 1 kg in Physics, the metric function $\rho$ here is not so naturally given as what we might keep in mind.

In mathematics, the set here can be any set $X$. It can be $\mathbb R^n$, a collection of some sets, or a of some functions.

And the metric function $\rho$ here can be any operation defined on $X \times X$, if it satisfies these conditions.

So, here we came up with the definition of metric spaces.


\begin{definition}
	
\end{definition}





\section{Bases of Sets}
%----------------------------------------


\begin{definition}
	\label{def: bases of sets}
	Let $X$ be a set, and let $\mathcal B \subseteq \mathcal P(X)$.
	
	$\mathcal B$ is a \textit{basis} of $X$ iff
	\begin{enumerate}
		\item 
		$\mathcal B$ is a cover of $X$, i.e., $X \subseteq \bigcup \mathcal B$; and
		
		\item
		For any $B_1, B_2 \in \mathcal B$, there exists a $\mathcal A \subseteq \mathcal B$, such that $B_1 \cap B_2 = \bigcup \mathcal A$.
	\end{enumerate}
\end{definition}


\begin{note}
	Some authors also call bases of sets \textit{synthetics sets}.
\end{note}



\section{Topological Spaces}
%----------------------------------------



\begin{definition}
	\label{def: topology}
	
	Let $X$ be any set.
	
	A collection $\mathcal T \subseteq \mathcal P(X)$ is a \textit{topology} for $X$ iff it satisfies the \textit{open set axioms}:
	\begin{enumerate}[(O1):]
		\item
		$X \in \mathcal T$;
		
		\item
		$\mathcal T$ is closed under arbitrary union; explicitly,
		$$
		\forall \mathcal A \subseteq \mathcal T: \bigcup \mathcal A \in \mathcal T;
		$$
		
		\item
		$\mathcal T$ is closed under finite intersection; explicitly,
		$$
		\forall \mathcal B \subseteq \mathcal T : |\mathcal B| \in \mathbb N : \bigcap \mathcal B \in \mathcal T.
		$$
	\end{enumerate}
	
	The ordered pair $(X, \mathcal T)$ is a \textit{topological space} iff $\mathcal T$ is a topology for $X$.
		
	A subset $U \subseteq X$ is an \textit{open set} of $(X, \mathcal T)$, or an \textit{open subset} of $X$, iff $U \in \mathcal T$.
\end{definition}


\begin{note}
	Even if $\mathcal T$ is an infinite	topology on an infinite set $X$, $\mathcal T$ is not needed to be closed under infinite intersection. For example, let $\mathcal T$
	$$
	\mathcal T = \left\{ [0,r): r \in \mathbb R \right\}.
	$$
	then $\mathcal T$ is a topology for $\mathbb R_{\ge 0}$. The collection
	$$
	\left\{ \left[ 0, \frac{1}{i} \right) \right\}_{i \in \mathbb Z_{> 0}}
	$$
	is a subset of $\mathcal T$, but its intersection is $\{0\} \notin \mathcal T$.
\end{note}



\begin{lemma}
	\label{lm: emptyset is in topology}
	
	Let $(X, \mathcal T)$ be a topological space.
	
	Then, $\emptyset \in \mathcal T$.
\end{lemma}


\begin{proof}
	As $\emptyset$ is a subset of any set, $\emptyset \subseteq \mathcal T$. By the open set axiom 2, we have
	$$
	\emptyset = \bigcup \emptyset \in \mathcal T.
	$$
\end{proof}


\begin{example}
	Let $X = \{1, 2, 3\}$, and let
	$$
	\mathcal B = \{\{1,2\}, \{2, 3\}, \{2\}\},
	$$
	and let $\mathcal T = \{\bigcup \mathcal A: \mathcal A \subseteq \mathcal B\}$, then $\mathcal T$ is a topology for $X$.
\end{example}


\begin{definition}
	Let $X$ be any set, and let $\mathcal T_1$ and $\mathcal T_2$ be topologies on $X$.
	
	$\mathcal T_1$ is said to be \textit{finer} than $\mathcal T_2$, or $\mathcal T_2$ is said to be \textit{coarser} than $\mathcal T_1$ iff $\mathcal T_2 \subseteq \mathcal T_1$.
\end{definition}


\begin{example}
	For any set $X$, the power set $\mathcal P(X)$ can be considered as a topology for $X$, called \textit{discrete topology}. It is the \textit{finest topology} on $X$.
\end{example}

\begin{example}
	For any set $X$, the collection $\{\emptyset, X\}$ is a topology for $X$. It is called \textit{indiscrete topology}, or \textit{trivial topology}, which is the coarsest topology on $X$.
\end{example}


\section{Interiors and Closures}
%----------------------------------------


\begin{definition}
	Let $(X, \mathcal T)$ be a topological space, and let $U \subseteq X$.
	
	The \textit{interior} of $A$, denoted $A^\circ$ or $\mathrm{int} (A)$, in $(X, \mathcal T)$ is defined as the union of all open sets contained in $A$. Explicitly, $\mathrm{int}_X$ can be considered as a mapping from $\mathcal P(X)$ to $\mathcal P(X)$, defined as
	$$
	\mathrm{int}_X(A) := \bigcup (\mathcal P(A) \cap \mathcal T).
	$$
\end{definition}


\begin{note}
	Finding the interior of a subset requires the definition of the topology for the set. I mean, even for the same set $X$ and the same subset $A \subseteq X$, if there are two different topologies $\mathcal T_1$ and $\mathcal T_2$ for $X$, the interior of $A$ in $(X, \mathcal T_1)$ and $(X, \mathcal T_2)$ might be different. For example, in $\mathbb R$, let $\mathcal T_1$ be indiscrete topology for $\mathbb R$, and let $\mathcal T_2$ be the Euclidean topology for $\mathbb R$, then,
	$$
	\mathrm{int}_{\mathcal T_1} ([0,1)) = \emptyset, \text{ and } \mathrm{int}_{\mathcal T_2}([0,1)) = (0,1),
	$$
	where $\mathrm{int}_{\mathcal T_1}(\cdot)$ and $\mathrm{int}_{\mathcal T_2}(\cdot)$ denotes the interior mapping for $(X, \mathcal T_1)$ and $(X, \mathcal T_2)$ respectively.
\end{note}


\begin{note}
	By the definition, it is clear that for any topology $(X, \mathcal T)$ and for any $A \subseteq X$, $\mathrm{int}(A) \in \mathcal T$.
\end{note}


\begin{lemma}
	Let $(X, \mathcal T)$ be a topological space, let $A \subseteq X$, and let $U \in \mathcal T$.
	
	Then, $U \subseteq A$ if and only if $U \subseteq \mathrm{int}(A)$.
	
	\begin{proof}
		As $U \in \mathcal P(A)$ and $U \in \mathcal T$, $U \in \mathcal P(A) \cap \mathcal T$. Thus,
		$$
		U \subseteq \bigcup \mathcal (\mathcal P(A) \cap \mathcal T) = \mathrm{int}(A).
		$$
		\qedlm
		
		Conversely, as $\mathrm{int}(A) \subseteq A$, as $U \subseteq \mathrm{int}(A)$, $U \subseteq A$.
	\end{proof}
\end{lemma}


\begin{lemma}
	Let $(X, \mathcal T)$ be a topological space, and let $A \subseteq X$.
	
	$\mathrm{int}(A) = A$ if and only if $A \in \mathcal T$.
	
	\begin{proof}	
		If $\mathrm{int}(A) = A$, then $A = \bigcup (\mathcal P (A) \cap\mathcal T)$. As $\mathcal P(A) \cap \mathcal T \subseteq \mathcal T$, this union is an element of $\mathcal T$.
		\qedlm
		
		Conversely, as $A \in \mathcal P(A)$ and $A \in \mathcal T$, $A \in \mathcal P(A) \cap \mathcal T$. For any $U \in \mathcal P(A)$, $U \subseteq A$. Then, we have $A \supseteq \bigcup (\mathcal P(A) \cap \mathcal T)$; and as $A \subseteq \bigcup (\mathcal P(A) \cap \mathcal T)$, we have
		$$
		A = \bigcup (\mathcal P(A) \cap \mathcal T) = \mathrm{int}(A).
		$$
	\end{proof}
\end{lemma}


\begin{lemma}
	Let $(X, \mathcal T)$ be a topological space, and let $A, B \subseteq X$.
	
	Then,
	$$
	\mathrm{int}(A \cap B) = \mathrm{int}(A) \cap \mathrm{int}(B).
	$$
	
	\begin{proof}
		Let $U \subseteq \mathrm{int}(A \cap B)$. Then,
		$$
		\begin{aligned}
			U \subseteq \mathrm{int}(A \cap B) = \bigcup (\mathcal P(A \cap \mathcal B) \cap \mathcal T).
		\end{aligned}
		$$
		
		Note that $\mathcal P(A \cap B) = \mathcal P(A) \cap \mathcal P(B)$, so $U \in \mathcal P(A \cap B) \cap \mathcal T$ iff $U \in \mathcal P(A) \cap \mathcal T$ and $\mathcal P(B) \cap \mathcal T$. We have
		$$
		\begin{aligned}
			U \subseteq \bigcup (\mathcal P(A \cap \mathcal B) \cap \mathcal T) &\iff U \subseteq \bigcup (\mathcal P(A) \cap \mathcal T) \land U \subseteq \bigcup (\mathcal P(B) \cap \mathcal T) \\
			&\iff U \subseteq \mathrm{int}(A) \land U \subseteq \mathrm{int}(B) \\
			&\iff U \subseteq \mathrm{int}(A) \cap \mathrm{int}(B).
		\end{aligned}
		$$
		
		Thus, $\mathrm{int}(A \cap B) = \mathrm{int}(A) \cap \mathrm{int}(B)$.
	\end{proof}
\end{lemma}



\section{Bases for Topologies}
%----------------------------------------


\begin{definition}
	Let $(X, \mathcal T)$ be a topological space.
	
	A collection $\mathcal B \subseteq \mathcal T$ is an \textit{analytic basis} for $\mathcal T$ iff for any $U \in \mathcal T$, there is an $\mathcal A \subseteq \mathcal B$, such that
	$$
	U = \bigcup \mathcal A.
	$$
\end{definition}


\begin{lemma}
	Let $(X, \mathcal T)$ be a topological space.
	
	A collection $\mathcal B \subseteq \mathcal T$ is an analytic basis for $\mathcal T$ iff for any $U \in \mathcal T$ and for any $x \in U$, there exists a $B \subseteq \mathcal B$, such that
	$$
	x \in B \subseteq \mathcal B.
	$$
\end{lemma}


\begin{proof}
	Let $\mathcal B$ be an analytic basis for $\mathcal T$.
	
	As $\mathcal B$ is a basis for $\mathcal T$, for any $U \in \mathcal T$, there exists a $B' \subseteq \mathcal B$ such that $U = \bigcup \mathcal B'$, which implies that for any $B' \in \mathcal B$, $B' \subseteq U$.
	
	$\qedlm$
	
	Conversely, let $\mathcal B \subseteq \mathcal T$ satisfies the condition after ``iff".
	
	Let $U \in \mathcal T$. For any $x \in U$, let $B_x \in \mathcal B$ with $x \in B_x \subseteq U$.
	
	As $\bigcup \{x\}_{x \in U} = U$, and $\{x\} \subseteq B_x$, we have
	
	$$
	U \subseteq \bigcup_{x \in U} B_x.
	$$
	
	As every $B_x \subseteq U$, we have
	$$
	\bigcup_{x \in U} B_x \subseteq U.
	$$
	Thus,
	$$
	U = \bigcup_{x \in U} B_x.
	$$
\end{proof}


=======


\begin{note}
	Explicitly, 2 can be considered as: for any $B_1, B_2 \in \mathcal B$, and for any $x \in B_1 \cap B_2$, there exists a $B_x \in \mathcal B$, such that
	$$
	x \in B_x \subseteq B_1 \cap B_2.
	$$
	(Why?)
\end{note}


\begin{note}
	$\emptyset$ is not necessary be an element of $\mathcal B$.
\end{note}


\begin{lemma}
	Let $(X, \mathcal T)$ be a topological space, and let $\mathcal B$ be an analytic basis for $\mathcal T$.
	
	Then, $\mathcal B$ is a synthetic basis of $X$.
\end{lemma}


\begin{proof}
	Let $B_1, B_2 \in \mathcal B$.

	As $\mathcal B$ is an analytic basis for $\mathcal T$, $\mathcal B \subseteq \mathcal T$, thus $B_1 \cap B_2 \in \mathcal T$.
	
	Thus, there exists an $\mathcal A \subseteq \mathcal B$, such that
	$$
	B_1 \cap B_2 = \bigcup \mathcal A.
	$$
	
	This precisely satisfies the definition of synthetic basis.
\end{proof}


\begin{lemma}
	Let $X$ be any set, and let $\mathcal B$ be a synthetic basis of $X$.
	
	Let
	$$
	\mathcal T = \left\{ \bigcup \mathcal A : \mathcal A \subseteq \mathcal B  \right\}.
	$$
	
	Then, $\mathcal T$ is a topology for $X$.
\end{lemma}


\begin{proof}
	As $\mathcal B$ is a synthetic basis of $X$, $X \subseteq \bigcup \mathcal B$. As $\mathcal B \subseteq \mathcal P(X)$, $\bigcup \mathcal B \subseteq X$. Thus, $X = \bigcup \mathcal B \in \mathcal T$.
	\qedlm

	Let $\mathcal U \subseteq \mathcal T$. For any $U \in \mathcal U$, there exists an $\mathcal A_U \subseteq \mathcal B$, such that $U = \bigcup \mathcal A_U$.
	
	We have
	$$
	\begin{aligned}
		\bigcup \mathcal U &= \bigcup \left\{ \bigcup \mathcal A_U \right\}_{U \in \mathcal U} \\
		&= \bigcup \left( \bigcup \left\{ \mathcal A_U \right\}_{U \in \mathcal U} \right)
	\end{aligned}
	$$
	
	As for any $U \in \mathcal U$, $\mathcal A_U \subseteq \mathcal B$, thus,
	$$
	\mathcal U = \bigcup \left\{ \mathcal A_U \right\}_{U \in \mathcal U} \subseteq \mathcal B.
	$$
	
	Thus, $\bigcup \mathcal U \in \mathcal T$. Therefore, $\mathcal T$ is closed under arbitrary union.
	\qedlm
	
	Let $\mathcal V$ be a finite subset of $\mathcal T$. For any $V \in \mathcal U$, there exists an $\mathcal A_V \subseteq \mathcal B$, such that $U = \mathcal A_V$.
	
	We have
	$$
	\begin{aligned}
		\bigcap \mathcal V &= \bigcap \left\{ \bigcup \mathcal A_V \right\}_{V \in \mathcal U} \\
		&= \bigcap \left( \bigcup \left\{ \mathcal A_V \right\}_{V \in \mathcal U} \right).
	\end{aligned}
	$$
	
	Similar to what we have proved above,
	$$
	\mathcal V = \bigcup \{\mathcal A_V\}_{V \in \mathcal V} \subseteq \mathcal B.
	$$
	
	Thus, $\bigcap \mathcal V \in \mathcal T$. Therefore, $\mathcal T$ is closed under finite intersection.
\end{proof}


\begin{lemma}
	Let $X$ be any set, and let $\mathcal C$ be a cover of $X$.
	
	The collection
	$$
	\mathcal B = \left\{ \bigcap \mathcal A: \mathcal A \subseteq \mathcal C \land |\mathcal A| \in \mathbb N \right\}
	$$
	is a synthetic basis of $X$.
\end{lemma}


\begin{proof}
	Let $B_1, B_2 \in \mathcal B$. There exist $\mathcal U, \mathcal V \subseteq \mathcal C$, such that $B_1 = \bigcup \mathcal U$ and $B_2 = \bigcup \mathcal V$. Then, we have
	$$
	\begin{aligned}
		B_1 \cap B_2 &= \bigcup_{U \in \mathcal U} U \cap \bigcup_{V \in \mathcal V} V \\
		&= \bigcup \{ U \cap V \}_{U \in \mathcal U, V \in \mathcal V}.
	\end{aligned}
	$$
	
	$\{U, V\} \subseteq \mathcal C$, so $U \cap V \in \mathcal B$. As $U$ and $V$ are arbitrarily taken from $\mathcal U$ and $\mathcal V$ respectively, $\{ U \cap V \}_{U \in \mathcal U, V \in \mathcal V} \subseteq \mathcal B$.
	
	Therefore, for any $B_1, B_2 \in \mathcal B$, there exists a finite $\mathcal A \subseteq \mathcal B$, such that $B_1 \cap B_2 = \bigcap \mathcal A$.
\end{proof}


\begin{note}
	In this note, we say that $\mathcal C$ \textit{generates} $\mathcal B$.
\end{note}


\begin{note}
	If $\mathcal C$ generates the synthetic basis $\mathcal B$, then $\mathcal B$ is the smallest synthetic basis containing $\mathcal C$. (Why?)
\end{note}


\begin{definition}
	Let $(X, \mathcal T)$ be a topological space, and let $\mathcal B$ be a synthetic basis of $X$.
	
	$\mathcal T$ is \textit{generated by} $\mathcal B$ iff
	$$
	\mathcal T = \left\{ \bigcup \mathcal A: \mathcal A \subseteq \mathcal B \right\}.
	$$
\end{definition}



\begin{example}
	In $\mathbb R^n$, for any $\mathbf x \in \mathbb R^n$, define
	$$
	B(\mathbf x, \delta) = \{ \mathbf y \in \mathbb R^n : \| \mathbf x - \mathbf y \| < \delta \land \delta \in \mathbb R_{> 0} \}.
	$$
	Let $\mathcal B$ be the set of all such $B(\mathbf x, \delta)$, then, $\mathcal B$ is a synthetic basis of $\mathbb R^n$, and it generates the \textit{Euclidean topology} for $X$.
\end{example}


\begin{example}
	In $\mathbb R^n$, let $\mathcal I$ be the set of all open intervals. $\mathcal I$ is a synthetic basis of $\mathbb R^n$, and it also generates the Euclidean topology for $\mathbb R^n$.
\end{example}


\begin{example}
	An \textit{ordered set} $(X, \preceq)$ is a set $X$ together with an \textit{ordering} $\preceq$ defined on $X$. That is, for any $x, y, z \in X$,
	\begin{enumerate}[(i)]
		\item
		(reflexive) $x \preceq x$;
		
		\item
		(transitive) $x \preceq y$ and $y \preceq z$ implies $x \prec z$;
		
		\item
		(antisymmetric) $x \preceq y$ and $y \preceq x$ implies $x = y$.
	\end{enumerate}
	
	$(X, \preceq)$ is an \textit{totally ordered set} iff $\preceq$ is \textit{connected}. That is, for any $x,y \in X$, $x \ne y$ implies $x\prec y$ or $y \prec x$.
	
	Now, let $(X, \preceq)$ be a totally ordered set, and let
	$$
	\mathcal A = \left\{X_{\prec x} : x \in X \right\} \cup \left\{ X_{\succ x} : x \in X \right\}.
	$$
	
	Let $\mathcal B$ be the synthetic basis generated by $\mathcal A$.
	
	Then, $\mathcal B$ generates an \textit{order topology} for $X$.
	
	If $\preceq$ is $\le$ on $\mathbb R$, then, the order topology for $\mathbb R$ is exactly the same as its Euclidean topology.
	
	Let
	$$
	\mathcal X = \left\{ \prod_{i = 1}^n \mathbb R_{< x_i} \right\} \cup \left\{ \prod_{i = 1}^n \mathbb R_{> x_i} \right\},
	$$
	let $\mathcal B$ be the synthetic basis for $\mathbb R^n$ generated by $\mathcal X$. Then $\mathcal B$ also generates the Euclidean topology for $\mathbb R^n$
\end{example}


\begin{note}
	In the example above, if $(X, \preceq)$ is an ordered set, but the connectedness of $\preceq$ is not required, then $\mathcal A$ is not a cover of $X$, and it generates no synthetic basis of $X$.
\end{note}


\begin{example}
	For any totally ordered set $(X, \preceq)$, the discrete topology for $X$ can be generated by either the collection of all closed intervals in $X$ or the collection of all singletons in $X$.
\end{example}


\begin{example}
	Let $(X, \preceq)$ be a totally ordered set, and let $C$ be a countable subset of $X$. The set
	$$
	\mathcal A = \{ X_{\prec x} : x \in C \}
	$$
	is a countable synthetic basis of $X$, and it generates a countable topology for $X$.
\end{example}


\begin{example}
	Let $X$ be an countably infinite set, and let $\mathcal B$ be the partition of $X$. As $\mathcal B$ is a synthetic basis for $X$, let $\mathcal T$ be the topology generated by $\mathcal B$.
	
	Then, $|\mathcal T| = |\mathcal P(\mathcal B)| = 2^{|\mathcal B|}$. Thus,
	\begin{enumerate}[(i)]
		\item $\mathcal T$ is finite iff $\mathcal B$ is finite;
		\item $\mathcal T$ is uncountable iff $\mathcal B$ is infinite (even if $\mathcal B$ is just countably infinite).
		\item $\mathcal T$ can not be countably infinite.
	\end{enumerate}
\end{example}



































%----------------------------------------
%########################################
%========================================
\end{document}